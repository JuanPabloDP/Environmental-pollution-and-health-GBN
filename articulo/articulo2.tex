\documentclass[twocolumn]{article}

\usepackage{graphicx} % Required for inserting images
\usepackage{amsmath}
\usepackage{amssymb}
\usepackage{url}
\usepackage[a4paper, left=1.5cm, right=1.5cm, top=2.5cm, bottom=2.5cm]{geometry}
\usepackage{etoolbox}
\usepackage{booktabs}

\patchcmd{\thebibliography}{\section*{\refname}}{}{}{}

\title{Gaussian Bayesian networks applied to an analysis of environmental pollution and health data in Mexico }

\author{
Alan Fernando Bravo Pimentel \and
Gonzalo Makenly Higuera Inzunza \and
Juan Pablo de la Peña Gonzalez \and
Sarah Camila Guzmán Fierro \and
Miguel Angel Rivas Torres
}

\date{Semptember 14th 2025}

\begin{document}

\maketitle

\begin{abstract}
% Resumen del artículo
\end{abstract}

\section{Introduction}
Climate change and air pollution are known to pose critical challenges when it comes to public health. In particular, exposition to gases such as nitrogen dioxide, ozone, and fine particulate matter have been linked to diminished biological biomarkers which are associated with cardiovascular risk, inflammation, and respiratory function. In Mexico, environmental pollution remains a significant health concern, and several measures have been taken in an attempt to mitigate the associated risks as much as possible. However, some of these dangers are not fully understood, and neither are the ways to counteract them.
Given that these contaminants interact with each other in intricate ways, complex relationships emerge which, in turn,  lend themselves to analysis via Gaussian Bayesian networks. 
In particular, 

%miguel pon tu investigación aquí, yo luego completo la intro más

\section{Methodology}



\section{Application}


\section{Conclusions}


\section{References}
\begin{thebibliography}{9}
\setlength{\itemsep}{0pt}
\setlength{\parskip}{0pt}

\bibitem{ref1} 



\end{thebibliography}

\end{document}
